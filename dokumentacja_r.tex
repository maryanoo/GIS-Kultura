%%%%%%%%%%%%%%%%%%%%%%%%%%%%%%%%%%%%%%%%%%%%%%%%%
%%%%%%%%%%%%%%%%%%%%%%%%%%%%%%%%%%%%%%%%%%%%%%%%%%%
% !TeX program = lualatex
\documentclass[a4paper,12pt,oneside,final]{mwrep}
\usepackage{polyglossia}
\setdefaultlanguage{polish}
\setmainfont{Iwona}
%%%%%%%%%%%%%%%%%%%%%%%%%%%%%%%%%%%%%%%%%%%%%%%%%%%%%%%%%
% Source: http://en.wikibooks.org/wiki/LaTeX/Hyperlinks %
%%%%%%%%%%%%%%%%%%%%%%%%%%%%%%%%%%%%%%%%%%%%%%%%%%%%%%%%%
\usepackage{graphicx}
\usepackage{booktabs}
\usepackage{tcolorbox}
\usepackage{lscape}
\usepackage{array}
\usepackage{chngpage}
\usepackage{subcaption} 
\captionsetup{compatibility=false}
\usepackage{caption}
\DeclareCaptionFont{white}{\color{white}}
\DeclareCaptionFormat{listing}{\colorbox{gray}{\parbox{\textwidth}{#1#2#3}}}
\captionsetup[lstlisting]{format=listing,labelfont=white,textfont=white}
\usepackage{listings}
\usepackage{underscore}
\usepackage{epstopdf}
\usepackage{longtable}
\usepackage[toc,page]{appendix}
\usepackage{pdflscape}
\usepackage{xcolor}
\lstset{basicstyle=\footnotesize\ttfamily,breaklines=true}
\lstset{framextopmargin=50pt,commentstyle=\itshape\color{purple!40!black}}
\lstset{
	numbers=left,
	stepnumber=1,    
	firstnumber=1,
	numberfirstline=false,
	escapeinside={\%*}{*)}
}
\lstset{extendedchars=true,inputencoding=utf8x,literate=%
{ą}{{\k{a}}}1
{Ą}{{\k{A}}}1
{ę}{{\k{e}}}1
{Ę}{{\k{E}}}1
{ó}{{\'o}}1
{Ó}{{\'O}}1
{ś}{{\'s}}1
{Ś}{{\'S}}1
{ł}{{\l{}}}1
{Ł}{{\L{}}}1
{ż}{{\.z}}1
{Ż}{{\.Z}}1
{ź}{{\'z}}1
{Ź}{{\'Z}}1
{ć}{{\'c}}1
{Ć}{{\'C}}1
{ń}{{\'n}}1
{Ń}{{\'N}}1
}

\usepackage{filecontents}
\begin{filecontents*}{Bibliografyr.bib}
@book{Lamigueiro2014,
	abstract = {data graphic is not only a static image, but it also tells a story about the data. It activates cognitive processes that are able to detect patterns and discover information not readily available with the raw data. This is particularly true for time series, spatial, and space-time datasets. Focusing on the exploration of data with visual methods, Displaying Time Series, Spatial, and Space-Time Data with R presents methods and R code for producing high-quality graphics of time series, spatial, and space-time data. Practical examples using real-world datasets help you understand how to apply the methods and code. The book illustrates how to display a dataset starting with an easy and direct approach and progressively adding improvements that involve more complexity. Each of the book's three parts is devoted to different types of data. In each part, the chapters are grouped according to the various visualization methods or data characteristics. Web Resource Along with the main graphics from the text, the author's website offers access to the datasets used in the examples as well as the full R code. This combination of freely available code and data enables you to practice with the methods and modify the code to suit your own needs.},
	address = {Madrid},
	author = {Lamigueiro, Oscar Perpi{\~{n}}{\'{a}}n},
	file = {:home/mariusz/Pobrane/spatial r books/(Chapman {\&} Hall{\_}CRC The R Series) Oscar Perpinan Lamigueiro-Displaying Time Series, Spatial, and Space-Time Data with R-Taylor and Francis, CRC Press (2014).pdf:pdf},
	isbn = {9781466565227},
	pages = {206},
	publisher = {Chapman {\&} Hall/CRC},
	title = {{Displaying Time Series, Spatial, and Space-Time Data with R}},
	year = {2014}
}
@book{Bivand2013,
	abstract = {Average Customer Rating: 5.0 Rating: 4 Overall good, but could be less technical I was really excited when I ordered this book as it looked like the type of material I had been looking for for ages, but as it turns out I am mildly disappointed in it, primarily because I found the text somewhat hard to grasp and the code not particularly well explained. It is still a very good read though and certainly helped me enhance my knowledge of statistical analyses in R. I would definitely recommend it to anyone looking to examine spatial data in R, but there is a bit of homework to do to be able to understand how all the pieces fit together. Rating: 5 Just what is needed This book fills a gap in the Spatial Statistics literature. Most of the treatises are heavy on the math, and I find it difficult to bridge the gap between the formulas and applying them. The modules for R take care of this for you and leave you to interpret the result. This book covers most of these modules and demonstrates how to use them. Really worth getting. There is also the ESRI Guide to GIS Analysis Vol 2, but it is more of an introductory text. Rating: 5 Excelent Book! Do you know that felling you may have when you found exactly what you were looking for? Well, that was the felling I had when started to read this book. It brings detailed information about how you can and should use the spatial data analysis resources of R, with interactive examples and enlighten explanations. You will really understand what you are doing and will find ways to represent it the best way you could. Rating: 5 Lots of information, clearly presented. This is not a book for a beginner, but is an excellent book for two groups of readers: those who have some background with R and wish to learn about its capabilities for spatial statistics; and those with some background in spatial statistics who wish to learn how to use R. The authors have been the main developers of the spatial statistics packages on R, and therefore know the packages intimately. But the authors also have a deep knowledge of the spatial statistics literature, and I found myself learning something new about these methods in every chapter. Everyone serious about spatial statistics should see what R has to offer. This book is the easiest way to do that. Rating: 5 The best practial introduction to R spatial The R spatial packages are the leading edge for spatial analysis and spatial statistics. This book, by the primary developers of the R Spatial packages, is the best introduction to the subject that I have seen. Now, if you are comfortable with it, you can dive an and download R and R spatial and go to town. But if you need some help, this is a good place to start. This would also make a good textbook for a class on spatial analysis.},
	address = {New York, Heidelberg, Dordrecht, London},
	author = {Bivand, Roger S and Pebesma, Edzer J and G{\'{o}}mez-Rubio, Virgilio},
	doi = {10.1007/978-1-4614-7618-4},
	file = {:home/mariusz/Pobrane/spatial r books/(Use R!) Roger S. Bivand, Edzer Pebesma, Virgilio G{\'{o}}mez-Rubio-Applied Spatial Data Analysis with R-Springer (2013).pdf:pdf},
	isbn = {978-1-4614-7618-4},
	pages = {405},
	publisher = {Springer},
	title = {{Applied spatial data analysis with R}},
	url = {http://link.springer.com/content/pdf/10.1007/978-1-4614-7618-4.pdf},
	year = {2013}
}
@inproceedings{Zielstra2015,
author = {Zielstra, Dennis and Tonini, Francesco},
booktitle = {North Carolina State University Geospatial Analytics Forum},
file = {:home/mariusz/Pobrane/spatial r books/Zielstra{\_}Tonini{\_}021215.pdf:pdf},
title = {{Analysis of Big Spatial Data with PostgreSQL / PostGIS and R – Case Studies in OpenStreetMap and Interactive Web Mapping from R PostgreSQL / PostGIS}},
year = {2015}
}
@book{Biecek2017,
	address = {Warszawa},
	author = {Biecek, Przemys{\l}aw},
	mendeley-groups = {Narzedzia R},
	publisher = {Oficyna Wydawnicza GiS},
	title = {{Przewodnik po pakiecie R}},
	year = {2017}
}
@book{biecek2012odkrywac,
	address = {Warszawa},
	author = {Biecek, Przemys{\l}aw},
	edition = {Drugie},
	isbn = {9788393969500},
	mendeley-groups = {Metodologia,Narzedzia R},
	publisher = {Fundacja Naukowa SmarterPoland.pl},
	title = {{Odkrywac! Ujawniac! Objasniac! Zbior esejow o sztuce prezentowania danych}},
	year = {2016}
}
 @Book{,
	author = {Hadley Wickham},
	title = {ggplot2: Elegant Graphics for Data Analysis},
	publisher = {Springer-Verlag New York},
	year = {2009},
	isbn = {978-0-387-98140-6},
	url = {http://ggplot2.org},
}
@book{Chang:2013:RGC:2484533,
	author = {Chang, Winston},
	title = {R Graphics Cookbook},
	year = {2013},
	isbn = {1449316956, 9781449316952},
	publisher = {O'Reilly Media, Inc.},
} 
@book{field2012discovering,
	title={Discovering Statistics Using R},
	author={Field, Andrew and Miles, Jeremy and Field, Zoe},
	isbn={9781446258460},
	year={2012},
	publisher={SAGE Publications}
}
@article{sorensen2002use,
	title={The use and misuse of the coefficient of variation in organizational demography research},
	author={S{\o}rensen, Jesper B},
	journal={Sociological methods \& research},
	volume={30},
	number={4},
	pages={475--491},
	year={2002},
	publisher={Sage Publications Thousand Oaks}
}

\end{filecontents*}
\usepackage{marginnote}
\usepackage[numbers]{natbib}
\bibliographystyle{plainnat}
\usepackage[
		pdfencoding=auto,% or unicode
		psdextra,
		]{hyperref}
\hypersetup{pdfinfo={
		Title={Analizy statystyczne i przestrzenne w R},
		Author={Mariusz Piotrowski},
		Subject={Skrypt przybliżający zagadnienia analiz statystycznych i korzystania z programu R do analiz danych ilościowych i danych przestrzennych.}
}}
%%%%%%%%%%%%%%%%%%%%%%%%%%%%%%%%%%%%%%%%%%%%%%%%%%%%%%%%%%%%%%%%%%%%%%%%%%%%%%%%
% 'dedication' environment: To add a dedication paragraph at the start of book %
% Source: http://www.tug.org/pipermail/texhax/2010-June/015184.html            %
%%%%%%%%%%%%%%%%%%%%%%%%%%%%%%%%%%%%%%%%%%%%%%%%%%%%%%%%%%%%%%%%%%%%%%%%%%%%%%%%
\newenvironment{dedication}
{
   \cleardoublepage
   \thispagestyle{empty}
   \vspace*{\stretch{1}}
   \hfill\begin{minipage}[t]{0.66\textwidth}
   \raggedright
}
{
   \end{minipage}
   \vspace*{\stretch{3}}
   \clearpage
}

%%%%%%%%%%%%%%%%%%%%%%%%%%%%%%%%%%%%%%%%%%%%%%%%
% Chapter quote at the start of chapter        %
% Source: http://tex.stackexchange.com/a/53380 %
%%%%%%%%%%%%%%%%%%%%%%%%%%%%%%%%%%%%%%%%%%%%%%%%
\makeatletter
\renewcommand{\@chapapp}{}% Not necessary...
\newenvironment{chapquote}[2][2em]
  {\setlength{\@tempdima}{#1}%
   \def\chapquote@author{#2}%
   \parshape 1 \@tempdima \dimexpr\textwidth-2\@tempdima\relax%
   \itshape}
  {\par\normalfont\hfill--\ \chapquote@author\hspace*{\@tempdima}\par\bigskip}
\makeatother

%%%%%%%%%%%%%%%%%%%%%%%%%%%%%%%%%%%%%%%%%%%%%%%%%%%
% First page of book which contains 'stuff' like: %
%  - Book title, subtitle                         %
%  - Book author name                             %
%%%%%%%%%%%%%%%%%%%%%%%%%%%%%%%%%%%%%%%%%%%%%%%%%%%

% Book's title and subtitle
\title{\Huge \textbf{Analizy statystyczne i przestrzenne w R}  \\ \huge Podstawy analiz i wizualizacji danych. }
% Author
\author{\textsc{dr Mariusz Piotrowski}}
\date{\today\\wersja robocza}

\begin{document}

%\frontmatter
\maketitle

%%%%%%%%%%%%%%%%%%%%%%%%%%%%%%%%%%%%%%%%%%%%%%%%%%%%%%%%%%%%%%%
% Add a dedication paragraph to dedicate your book to someone %
%%%%%%%%%%%%%%%%%%%%%%%%%%%%%%%%%%%%%%%%%%%%%%%%%%%%%%%%%%%%%%%


%%%%%%%%%%%%%%%%%%%%%%%%%%%%%%%%%%%%%%%%%%%%%%%%%%%%%%%%%%%%%%%%%%%%%%%%
% Auto-generated table of contents, list of figures and list of tables %
%%%%%%%%%%%%%%%%%%%%%%%%%%%%%%%%%%%%%%%%%%%%%%%%%%%%%%%%%%%%%%%%%%%%%%%%
\tableofcontents
\lstlistoflistings

% Preface %
%%%%%%%%%%%
\chapter{Informacje wstępne.}

Kurs podstawowy R można rozpocząć od zainstalowania pakietu w programie R:
\begin{lstlisting}[language=R,caption={Interaktywny kurs R - instalacja i uruchomienie. }]
install.packages("swirl")
library("swirl")
swirl()
\end{lstlisting}
Dodatkowe kursy można doinstalować później. Znajdują się one na stronie \url{https://github.com/swirldev/swirl_courses}.

Zagadnienia bazują na różnorodnych materiałach. Podstawowe strony z analizą przestrzenną to:
\begin{itemize}
\item \url{http://spatial-analyst.net/wiki/index.php?title=Software}
\item \url{http://pakillo.github.io/R-GIS-tutorial/}
\item \url{http://spatial.ly/r/}
\item \url{http://oscarperpinan.github.io/spacetime-vis/}
\end{itemize}
oraz \url{https://www.r-bloggers.com/the-guerilla-guide-to-r/}



\chapter{Konfiguracja środowiska R.}

\chapter{Połączenie z bazą Postgresql.}

\chapter{Typy danych.}

Dane w R przechowywane mogą być w formie:
\begin{description}
	\item[wektora wartości] - kolekcja wartości (np. 1,2,3)należące do tej samej klasy. Mogą to być:
	\begin{itemize}
		\item \textbf{[num]} - wartości liczbowe z wartościami dziesiętnymi
		\item \textbf{[int]} - liczby całkowite - bez wartości dziesiętnych
		\item \textbf{[logi]} - operator logiczny - prawda/fałsz
		\item \textbf{[factor]} - zmienne jakościowe (jeśli są uporządkowane wówczas mają klase \textbf{[ordered]})
	\end{itemize}
	\item[time.series -ts] - wektor + zmienna z informacją o dacie
	\item[data.frame] -(ramka danych) dane przechowywane w układzie: \begin{itemize}
		\item kolumny - zmienne
		\item wiersze - obserwacje
	\end{itemize}
	\item[listy] - struktura, która pozwala na zagnieżdżanie w niej innych elementów - np. ramek danych, czy innych list.
\end{description}
\section{Wektor wartości.}
Najprostszy sposób tworzenia wektora wartości to użycie funkcji \textit{concatenate} /połącz/ (c).
\begin{lstlisting}[language=R,caption={Tworzenie wektora danych funkcją \textit{concatenate} c(). }]
 x <- c(1,2,3,4)
\end{lstlisting}

\section[Indeksowanie danych]{Odwoływanie się do wartości}

Indeksowanie danych, czyli odwoływanie się do określonych wartości z wektora, lub ramki danych - odbywa się przez użycie nawiasu kwadratowego [\,]. 
\paragraph{Wektor wartości}
Dla wektora [\,] określa pozycję wartości - np:
\begin{lstlisting}[language=R,caption={ Wskazanie konkretynych wartości z wektora.}]
y <- x[c(2,3)] 
\end{lstlisting}
Mogą być użyte ciągi liczbowe:
\begin{lstlisting}[language=R,caption={Wskazanie ciągu wartości z wektora. }]
y <- x[2:4]
\end{lstlisting}
Mogą być używane operatory logiczne:
\begin{lstlisting}[language=R,caption={Wskazanie wartości przez użycie operatora logicznego. }]
y <- x[x > 3]
\end{lstlisting}
\paragraph{Ramki danych}
Dla ramek danych podaje się indeks wiersza, następnie indeks kolumny.
W pakiecie R wbudowana jest ramka danych - mtcars. Kolejny przykład odwołuje się do tych danych. 
\begin{lstlisting}[language=R,caption={Wskazanie wartości z ramki danych - ciąg danych. }]
z <- mtcars[,1:3]
\end{lstlisting}
Efektem będzie użycie wszystkich wierszy i kolumn od 1 do 3. 
W ramach indeksu można filtrować dane używając operatorów logicznych.
\begin{lstlisting}[language=R,caption={Wskazanie konkretnych wartości z ramki danych.}]
 mtcars[mtcars[, "mpg"] > 21,]
\end{lstlisting}
Funkcja pokaże tylko wiersze, które spełniają warunek - w kolumnie mpg wartości sa większe od 21. Pokazane są w wyniku wszystkie kolumny.\par
W przypadku, kiedy chcemy odwołać się do jednej kolumny ( i uzyskać zamiast ramki danych - wektor) należy użyć znaku \$\ .\footnote{W RStudio po wpisaniu znaku \$\ pojawią się podpowiedzi z nazwami kolumn.}
 
 \begin{lstlisting}[language=R,caption={Wybór jednej kolumny. Wyodrębnienie wektora wartości z ramki danych. }]
 mtcars$mpg
 \end{lstlisting}
\paragraph{Dodawanie nowej kolumny w ramce danych.} Używając funkcji \$\ można dodawać nowe kolumny.
\begin{lstlisting}[language=R,caption={Dodawanie nowej kolumny. }]
mtcars$mpgtest <- mtcars$mpg * 2
\end{lstlisting}
W efekcie powstanie nowa kolumna mpgtest,której wartości są pomnożonymi razy 2 wartościami z kolumny mpg.

\paragraph{Praca na danych sondażowych - case studies Uczestnictwo w kulturze - Katowice}

Zbudowanie prostej tabel krzyżowej, zgrupowania zmiennych wg jakichś cech - wymaga odwołania się do indeksu. \par
Z ramki danych (survey-data) chcę zaprezentować dane o wydarzeniu i poziomie wykształcenia uczestników. [,c(...)] oznacza - wybierz WSZYSTKIE wiersze.
\begin{lstlisting}[language=R,caption={Identyfikacja numeru i tworzenie ramki danych z dwóch zmiennych. }]
which(colnames(survey.data)=="event")
which(colnames(survey.data)=="education_level")
x <- survey.data[,c(1,13)]
\end{lstlisting}

Do analizy danych sondażowych, czyli wszędzie tam, gdzie głównie analizuje się dane mierzone na skalach jakościowych można użyć pakiet \,<<questionr>>. Przydatne są szczególnie funkcje do analiz procentów i pokazywania procentów w tabelach krzyżowych.
\begin{lstlisting}[language=R,caption={Liczenie procentów ze zmiennej. }]
library(questionr)
freq(survey.data$education_level)
\end{lstlisting}
\begin{lstlisting}[language=R,caption={Liczenie procentów w tabeli krzyżowej. }]
library(questionr)
x <- survey.data[,c(1,13)]
#zrobienie ramki danych z dwoma zmiennymi
y <- table(x)
#zrobienie tabeli
rprop(y)
#tabela krzyżowa z procentami dla wierszy
cprop(y)
#tabelea krzyżowa z procentami dla kolumn
\end{lstlisting}


\chapter{Podstawowe operacje statystyczne.}

\section{Opis jednej zmiennej}

Zebrane dane z badań porządkowane są w szeregi statystyczne, dzięki czemu możliwe jest określenie rozkładu wartości, które przyjmuje każda z badanych zmiennych. W tym celu stosuje się różnego rodzaju miary. W celu określenia tego co jest typowe dla zmiennej stosuje się miary skupienia, a w celu ustania wewnętrznej różnorodności zmiennej stosuje się miary rozproszenia.
\paragraph{Statystyki opisowe w R} - W przypadku zmienny ilościowych [numeric] podstawowe statystyki można uzyskać funkcją - \textit{summary()}.\par
Odpowiednikiem jest tzw. pięć liczb Turkeya - \textit{fivenum()}:
\begin{itemize}
	\item wartość minimalna
	\item granica pierwszego kwartyla
	\item mediana
	\item średnia (tylko w funkcji \textit{summary()})
	\item granica trzeciego kwartyla
	\item wartość maksymalna
\end{itemize}
W przypadku zmiennych jakościowych funkcja \textit{summary()} pokazuje tablicę częstości zmiennej (uwzględniane są braki danych). Dzięki funkcji \textit{table()} można zbudować tabele kontyngencji (bez braków danych).
\begin{lstlisting}[language=R,caption={Podsumowanie statystyk opisowych}]
summary(mtcars$mpg)
\end{lstlisting}

\paragraph[Miary skupienia]{Miary tendencji centralnej}



\subparagraph{Wielkości średnie}

Najpopularniejszą miarą skupienia jest średnia arytmetyczna. Mierzy się go jako iloraz sumy wartości pomiarów przez ich liczbę \[  \bar{X} = \frac{\sum x_{i}}{N}  \]. \par
W przypadku obliczania \textbf{średniej ważonej} każdy pomiar jest mnożony przez wagę, następnie wynik dzielony jest przez liczbę pomiarów. W R średnie można obliczyć następująco:
\begin{lstlisting}[language=R,caption={Obiczanie średnich}]
mean(dane$zmienna)
weighted.mean(dane$zmienna, dane$wagi)
1/mean(1/a) #compute the harmonic mean
\end{lstlisting}

W badaniach społecznych może pojawić się potrzeba użycia innego rodzaju średnich.Warto zwrócić uwagę na: \begin{itemize}
	\item średnią odciętą (trymowaną) wyliczana jest średnia odejmując 5\% dolnych i 5\% górnych wyników
	\item  średnia geometryczna (Wskaźniki w HDI są tak obliczane).  Stosowana do określenia przeciętnej wielkości jakiejś zmiany zachodzącej w badanym środowisku. Ale zmiana ma charakter względnie regularny. Np. jakieś zwiększenie cechy rośnie w postępie geometrycznym \[ G = \sqrt[n]{x_{1} x_{2}... x_{n}}\] i wszystkie $ x_{i} >0 $
	\item średnia harmoniczna (wykorzystywanej przy obliczeniu średniej liczby mieszkańców na $km^{2})$ i \[ H =  \]
	\item średnia krocząca (średnia ruchoma)
	
	
\end{itemize} .

\subparagraph{Wartość środkowa}

Obliczanie mediany w R 

\begin{lstlisting}[language=R,caption={Obiczanie wartości środkowej}]
median(dane$zmienna)
\end{lstlisting}

\paragraph{Miary rozproszenia - dyspersji}

\subparagraph{Odchylenie średnie, przeciętne}
Miara praktycznie już nie stosowana w statystyce. 
Odchylenie przeciętne to średnia arytmetyczna wartości bezwzględnych (absolutnych, czyi pomijając znak przed wartością) wszystkich odchyleń poszczególnych  wartości pomiarowych od ich  średniej arytmetycznej. Uproszczony wzór to \[  d = \frac{\sum |x_{i} - \bar{x}|}{n} \] \par

W R do obliczenia można użyć formuły.
\begin{lstlisting}[language=R,caption={Obliczanie odchylenia przeciętnego }]
mean(abs(dane$zmienna-mean(dane$zmienna)))
\end{lstlisting}

\subparagraph{Wariancja i odchylenie standardowe}

Wariancja jest obliczania podobnie jak odchylenie przeciętne, jednak zamiast wartości bezwzględnej, natomiast mianownik to liczba obserwacji pomniejszony o 1. (w przypadku obliczeń dla próby) Wzór to
\[  s^{2} = \frac{\sum (x_{i} - \bar{x})^{2}}{n-1} \] \par
Odchylenie standardowe jest pierwiastkiem kwadratowym z wariancji. Wzór to 
\[  s = \sqrt{\frac{\sum (x_{i} - \bar{x})^{2}}{n-1}} \] \par
W R oblicza się ją jako 
\begin{lstlisting}[language=R,caption={obliczanie wariancji i odchylenia standardowego  }]
var(dane$zmienna)
sd(dane$zmienna)
\end{lstlisting}

Wadą odchylenia standardowego jest silna podatność na wartości skrajne danej zmiennej.

\subparagraph{Współczynnik zmienności - Coefficent of variation }

Budzącym kontrowersje parametrem określającym rozproszenie jest współczynnik zmienności Pearsona. Zaletą jego jest łatwe obliczenie, oraz zastosowanie do porównań pomiędzy grupami. Do wyliczenia stosuje się iloraz odchylenia standardowego i średniej arytmetycznej  \[  V = \frac{s}{\bar{x}}  \] i  $ \bar{x} \neg 0  $. Wartość współczynnika można podać w procentach
\[  V_{zmiennej} = \frac{s}{\bar{x}} *100  \]  = \% . O kontrowersjach w użyciu tego współczynnika można poczytać u \citeauthor{sorensen2002use}.\footnote{\url{https://web.stanford.edu/~sorensen/nomorecv\%20revision\%20final.pdf}}. Aby obliczyć ten współczynnik w R należy zastosować taką formułę:
\begin{lstlisting}[language=R,caption={Obliczanie współczynnika zmienności}]
sd(Data$Variable, na.rm=TRUE)/
	mean(Data$Variable, na.rm=TRUE)*100
\end{lstlisting}


\section{Opis związku dwóch zmiennych. Zmienne nominalne.}

Jakość ustalonej skali rozstrzyga o tym, jakie operacje badawcze moją być wykonane. Jeśli obie zmienne wyrażone są na skalach ilościowych, lub porządkowych wówczas mowa o korelacji, jeśli wyrażone są na skalach nominalnych (jakościowych) mowa o zbieżności, czy asocjacji.\par

Co ważne - wielkość związku zmiennych w próbie, jedynie określa nam, czy istnieje, czy też nie związek między zmiennymi, nie mówi nam czy przekłada się on na populacje, nie przekłada się on na statystyczną istotność. Test $\chi^2$ określa właśnie to, czy można przenieść wnioski na populację - czy związek jest statystycznie istotny, czy też nie.

\begin{tabular}{|p{2,2cm}|p{3cm}|p{3cm}|p{4cm}|}
	\hline 

	 & ilościowa & porządkowa & nominalna  \\ 
	\hline 
	 ilościowa & Pearsona współczynnik korelacji (r) &  &  \\ 
	\hline 
	 porządkowa &  & \begin{itemize}
	 	\item Spearmana współczynnik korelacji rangowej/pozycyjnej (R)
	 	\item $\tau$ (tau) Kendalla
	 	\item Kendalla współczynnik zgodności
	 	\item $\gamma$ (gamma Goodmana i Kruskala)
	 \end{itemize} &  \\ 
	\hline 
	 nominalna & \begin{itemize}
	 	\item stosunek korelacyjny $\eta$ (eta) 
	 	\item współczynnik korelacji dwuseryjnej punktowej
	 \end{itemize} &  & \begin{itemize}
	 	\item współczynnik zbieżności/kontyngencji (C),
	 	\item współczynnik asocjacji (Q) Yula, 
	 	\item współczynnik $\phi$ (phi) (dla cech zdychotomizowanych)
	 	\item V Cramera
	 	\item $\lambda$ (lambda) Goodmana i Kruskala
	 \end{itemize} \\ 
	\hline 
 
\end{tabular} 

\subsection{Współczynnik asocjacji Q Yule`a}

Najprostszym sposobem do obliczeń asocjacji jest współczynnik asocjacji Q - Yule`a (Yule`a - Kendalla), który można stosować wyłącznie do tabel dwudzielnych (2x2), ale w takiej tablicy nie powinny być wartości 0, gdyż współczynnik Q będzie 1, lub -1. \par
Do obliczeń stosuje się wzór:
\[ Q = \frac{ad - bc}{ad - bc} \]

\begin{tabular}{|c|c|c|c|}
	\hline 
	& Wyrzucanie śmieci  & Śmiecenie  & Suma  \\ 
	\hline 
	Kobiety	& 18 (a) & 7 (b) & 25 (a+b) \\ 
	\hline 
	Mężczyźni	& 42 (c) & 33 (d)  & 75 (c+d)  \\ 
	\hline 
	Suma	& 60 (a+c)  & 40 (b+d) & 100 (N) \\ 
	\hline 
\end{tabular} \par


W efekcie dla tej tabeli dwudzielnej wynik wynosi $ Q = 0.33 $, czyli jest to związek słaby. Q może przyjmować wartości od -1 do +1. Dodatnie wartości świadczą, że I wariant cechy x współwystępuje z I cechą y, a II wariant cechy x z II wariantem cechy y. Ujemne wartości oznaczają, że I wariant cechy x kojarzy się z II wariantem cechy y, zaś II wariant cechy x, z I wariantem cechy y.	`

\subsection{ $\chi^2$ (Chi kwadrat) - test niezależności zmiennych}
Dla zmiennych jakościowych -kategorialnych lub nominalnych możliwe jest określenie różnicy w rozkładzie zmiennej w badanej próbie. Odbywa się to przez przyjęcie, lub odrzucenie hipotezy zerowej $H_{0} $ - w brzmieniu - nie ma statystycznej różnicy w rozkładzie cech zmiennej. 
$\chi^2$ pozwala na określenie, czy dane w próbie (rozkład według kategorii zmiennej nominalnej) wyniki rozłożyły się wedle proporcji, które są przypadkowe, czy też nie.\par 
Dla małych prób, oprócz wykonania testu na nieciągłość przy obliczaniu $\chi^2$ , należy wykonać test Fishera aby uniknąć błędu z odrzuceniem hipotezy. 
Test dokładny Fishera przeprowadza się dla bardzo małych prób (np. gdy jedna z liczebności w komórek jest < (mniejsza niż 5) i tablic 2x2. Określa on dokładne prawdopodobieństwo, a nie przybliżone. Test $\chi^2$   może podać prawdopodobieństwo pozwalające na odrzucenie hipotezy zerowej, zaś Test Fishera sprzyja nieodrzuceniu hipotezy zerowej. 

\par 

Przykładem może być rozkład cechy wyrzucanie śmieci a płeć. Dane prezentuje tabela

\begin{tabular}{|c|c|c|c|}
	\hline 
	& Wyrzucanie śmieci  & Śmiecenie  & Suma  \\ 
	\hline 
	Kobiety	& 18 (a) & 7 (b) & 25 (a+b) \\ 
	\hline 
	Mężczyźni	& 42 (c) & 33 (d)  & 75 (c+d)  \\ 
	\hline 
	Suma	& 60 (a+c)  & 40 (b+d) & 100 (N) \\ 
	\hline 
\end{tabular} 

Pierwszym krokiem jest określenie wartości oczekiwanych dla rozkładu cechy. Zasada brzmi: \textit{suma wiersza pomnożona przez sumę kolumny, podzielona przez sumę ogólną.} Można to zrobić wg formuły. Wartość oczekiwana dla komórki a:

\[ a_{oczekwiana} = \frac{(a+b) * (a+c)}{N} \], czyli \[ a_{oczekiwana} = \frac{25 * 60}{100} = 15\]

a nowa tabela będzie wyglądała następująco

\begin{tabular}{|c|c|c|c|}
	\hline 
	& Wyrzucanie śmieci  & Śmiecenie  & Suma  \\ 
	\hline 
	Kobiety	& 18 (15) & 7 (10) & 25 (a+b) \\ 
	\hline 
	Mężczyźni	& 42 (45) & 33 (30)  & 75 (c+d)  \\ 
	\hline 
	Suma	& 60 (a+c)  & 40 (b+d) & 100 (N) \\ 
	\hline
\end{tabular} \par
Wzór na chi kwadrat prezentuje się następująco:
\[\chi^2=\sum_{k=1}^{n} \frac{(O_k - E_k)^2}{E_k}\]
gdzie O to wartość obserwowana, a E to wartość oczekiwana.  \par
Dla powyższego przypadku (i prostego zastosowaniu wzoru) $ \chi^{2} $ = 2.0. Po sprawdzeniu w tabeli - dla 1 stopnia swobody\footnote{Stopień swobody oblicza się jako iloczyn liczby kolumn -1 i liczy wierszy -1 (k-1)(w-1)} - prawdopodobieństwo wynosi 0.16 (więc jest (większe) > od $\alpha$ = 0,05 lub $\alpha$ = 0,01 ) - więc nie odrzucamy $ H_0 $. Nie ma istotnej różnicy w sposobie postępowania ze śmieciami ze względu na płeć.  \par
Jeśli p < $\alpha$ = 0,05 lub $\alpha$ = 0,01 (czyli jest mniejsze) wówczas hipotezę o niezależności odrzucamy.\par
W R, aby obliczyć chi kwadrat wystarczy zastosować test.
\begin{lstlisting}[language=R,caption={Testowanie chi kwadrat. }] 
s <-smieci[,c(2,3)]
#stworzenie tablicy - indeksowanie
chisq.test(s$wyrzucanie.smieci, s$X.1)
#przprowadzenie testu 
\end{lstlisting}

Dla powyższych danych R automatycznie dołącza poprawkę Yates'a ( dla wartości, które są mniejsze niż 10). Dodatkowo może pojawić się potrzeba przeprowadzenia dokładnego testu Fishera. \par
\begin{tcolorbox}
	
	
	Pearson's Chi-squared test with Yates' continuity correction \par
	
	data:  s$wyrzucanie.smieci and s$X.1 \par
	X-squared = 1.3889, df = 1, p-value = 0.2386 \par
\end{tcolorbox}



\subsection{Współczynni phi \phi}

Dla zmiennych uszeregowanych podwójnie(dychotomicznie), kiedy zmienna zawiera tylko dwie klasy - obliczyć można siłę związku obliczając współczynnik \phi (phi). Problem z tym współczynnikiem polega na tym, że jeśli tabela jest większa to wartość może być większa niż 1, co utrudnia proces interpretacji współczynnika.

\[ \phi = \frac{ad - bc}{\sqrt{(a+b)(c+d)(a+c)(b+d)}} \]

Między \phi i $\chi^2$ istnieje związek bezpośredni \[ \phi^2 = \frac{\chi^2 }{N} \] stąd \[ \phi = \sqrt{\frac{\chi^2}{N}} \]
Sprawia to, że przed określeniem siły związku między zmiennymi w tabeli należy określić, czy jest to związek nie wynikający z przypadku przeprowadzając test na niezależność zmiennych $ \chi^2 $. Wzór w tej formie może być używany dla tabel większych niż 2x2.\par
Znak przy \phi nie określa kierunku związku, jak w przypadku korelacji miarowej(Pearsona), jedynie zależy do uporządkowania danych w tabeli 2x2.

\subsection{Współczynnik kontygencji C - Pearsona}

Współczynnik kontygencji w przeciwieństwie do współczynnika $\phi$ (phi), czy Q Yule`a można stosować dla tabel wielopolowych, bez ograniczenia do wielkości 2x2 (tabeli czteropolowej). Wartości, które przyjmuje współczynnik kontygencji C są z zakresu od 0 do 1. 

\[ C = \sqrt{\frac{\chi^2}{\chi^2 + N}} \]

\subsection{V Cramera }

Kiedy współczynnik jest wyliczany dla tabeli 2x2 jego wartość jest taka jak we współczynniku kontygencji C. Zastąpił on używany wcześniej współczynnik Czuprowa T

\[ V = \sqrt{\frac{\chi^2}{(m-1)N}} \]
m to liczba kolumn, lub wierszy w zależności od tego, która wielkość jest mniejsza


\subsection{Współczynnik lambda $\lambda$ Goodmana i Kruskala }

Współczynnik bazuje na koncepcji proporcjonalnej redukcji błędu. Zawiera się w przedziale 0 do 1. Miarę tą można obliczać jako: \begin{itemize}
	\item symetryczną, wówczas nie ma znaczenia która zmienna jest zależna, a która niezależna
	\item niesymetryczną, wówczas test jest wykonywany osobno, więc zmiana kolejności będzie skutkowała innymi wynikami
\end{itemize}  

\section{Opis związku dwóch zmiennych. Zmienne porządkowe.}

\subsection{Współczynnik $ \gamma$ (gamma) Goodmana i Kruskala}

W pakiecie SPSS można go używać do wyliczania współczynnika Q Yula. 

\subsection{Współczynnik $\rho$ (rho)  Spearmana. Korelacji rang}

Współczynnik używany, kiedy dane są porangowane.  Porównujemy uszeregowanie dwóch zbiorów danych: obliczamy różnicę rang, podnosimy je do kwadratu, sumujemy. Wartość tego współczynnika zawiera się w przedziale  -1 < 0 < 1. 

\subsection{Współczynnik $\tau$ (tau) b Kendalla. Korelacji rang }

Współczynnik używany, kiedy dane porządkowe są mają powiązane rangi.


\section{Opis związku dwóch zmiennych. Zmienne ilościowe.}
\subsection{Korelacja miarowa/liniowa Pearsona} - Tworzenie macierzy korelacji odbywa się następująco:
\begin{lstlisting}[language=R,caption={Korelacja}]
cor(mtcars[,1:5])
#lub wybór konkretnych kolumn 
cor(mtcars[,c(1,4)])
\end{lstlisting}

\chapter{Tworzenie wykresów.}
\section{Pakiet podstawowy.}
Prosty mechanizm wizualizacji danych pozwalających na prezentacje statystyk.
\subsection{Wykres słupkowy}
Pozwala na prezentację jednej, lub dwóch zmiennych kategorialnych. Do rysowania wykresu słupkowego służy funkcja \textit{barplot()}. \paragraph*{Częstości} sa generowane funkcją \textit{table()}.
\begin{lstlisting}[language=R,caption={Rysowanie wykresu słupkowego. }]
tab <- table(mtcars$cyl)
barplot(tab, horiz = FALSE, las = 1)
\end{lstlisting}

\subsection{Wykres pudełkowy.}

\subsection{Wykres kropkowy.}
\section{ Pakiet ggplot2.}
Tutorial \url{http://r-statistics.co/ggplot2-Tutorial-With-R.html}

\chapter{Wizualizacja danych w ggplot2.}

Do wizualizacji danych został wykorzystany pakiet ggplot z pakietu R. Podstawowe operacje "czyszczące" i porządkujące dane zostały wykonane w arkuszu kalkulacyjnym, zapisane w formacie csv i zaimportowane do programu RStudio.
\section{Wykres słupkowy - porównanie danych.}
\begin{itemize}
	\item Rysowanie wykresu w oparciu o przekształcenie danych krótkich "short form" w długie "long form".
	\item Użycie danych surowych <<stat="identity">>
	\item Obrót wykresu.
\end{itemize}

\begin{lstlisting}[ language=R]
library(reshape)    
library(ggplot2)
dataframe <- melt(krs_osm )
dataframe  <- dataframe [complete.cases(krs_osm ),]
head(dataframe)
ggplot(data = dataframe , aes(x=reorder(Nazwa_indeksu,value), y = value, fill = variable), summarise ) + geom_bar(stat="identity", width = 0.5) + 
labs (y="Liczba obiektow", x="Nazwa indeksu infrastruktury", size =2, fill='Zrodlo danych' )  +
geom_text(aes(label=value), position=position_dodge(width= 0.1 ),hjust = - 0.1, vjust= 0.4,size = 3, color = "black"  ) + expand_limits(y=c(0,77000))+
coord_flip()
\end{lstlisting}

\section{Wykres skrzypcowy i pudełkowy.}

\begin{lstlisting}[ language=R]
library(ggplot2)
ggplot(wydatki_osoba_hist, aes(x=typ_historyczny, y= wydatki)) +
geom_violin(aes(fill=typ_historyczny), trim=F) + geom_boxplot(width=.2) + xlab("Typ historyczny") + ylab("Wydatki na osobe") +
geom_text(aes(label=gmina),color="blue3", size=3) +
geom_rug(sides = 'l')
\end{lstlisting}

\chapter{Tworzenie map. Pakiet ggplot2.}
Kurs online \url{https://www.datacamp.com/courses/working-with-geospatial-data-in-r}
Mapa świata \url{https://www.r-bloggers.com/how-to-make-a-global-map-in-r-step-by-step/}

\nocite{*}
\bibliography{Bibliografyr}
\end{document}