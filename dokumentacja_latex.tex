
% !TeX program = lualatex
\documentclass[a4paper,11pt,oneside]{mwrep}
\usepackage{polyglossia}
\setdefaultlanguage{polish}
\defaultfontfeatures{Ligatures=TeX}
\setmainfont{Iwona}
%\setmainfont{TeX Gyre Heros}

\usepackage{graphicx}
\usepackage{import}
\usepackage{booktabs}
\usepackage{lscape}
\usepackage{array}
\usepackage{chngpage}
\usepackage{subcaption} 
\captionsetup{compatibility=false}
\usepackage{listings}
\usepackage{underscore}
\usepackage{epstopdf}
\usepackage{longtable}
\usepackage{pdflscape}
\usepackage{xcolor}
\lstset{basicstyle=\footnotesize\ttfamily,breaklines=true}
\lstset{framextopmargin=50pt,commentstyle=\itshape\color{purple!40!black}}
\lstset{
	numbers=left,
	stepnumber=1,    
	firstnumber=1,
	numberfirstline=false
}
\usepackage[
		pdfencoding=auto,% or unicode
		psdextra,
		]{hyperref}
\hypersetup{pdfinfo={
		Title={Dokumentacja Latex},
		Author={Mariusz Piotrowski}
}}
\usepackage{tcolorbox}
%%%%%%%%%%%%%%%%%%%%%%%%%%%%%%%%%%%%%%%%%%%%%%%%%%%

% Book's title and subtitle
\title{\Huge \textbf{Skład raportów w \LaTeX}  \\ \huge Zarządzanie dokumentacją i tekstem. }
% Author
\author{\textsc{dr Mariusz Piotrowski}}
\date{\today\\wersja robocza}

\begin{document}

%\frontmatter
\maketitle


%%%%%%%%%%%%%%%%%%%%%%%%%%%%%%%%%%%%%%%%%%%%%%%%%%%%%%%%%%%%%%%%%%%%%%%%
% Auto-generated table of contents, list of figures and list of tables %
%%%%%%%%%%%%%%%%%%%%%%%%%%%%%%%%%%%%%%%%%%%%%%%%%%%%%%%%%%%%%%%%%%%%%%%%
\tableofcontents

%\mainmatter

%%%%%%%%%%%
% Preface %
%%%%%%%%%%%
\chapter{Informacje wstępne}

Dokument ma charakter informacji o sposobach budowania raportów badawczych i dokumentacji technicznej przy użyciu \LaTeX w systemie Linux. Zaprezentowane zostaną listingi kodu, które bazują na modyfikowanej klasie: book.\par
Wszystkie przykłady zostały przetestowane w środowisku Linux (Arch)\footnote{\url{https://www.archlinux.org/}}, które wykorzystuje dystrybucję TeXLive. Pakiety \LaTeX pochodzą z metapakietu texlive-most.\footnote{\url{https://www.archlinux.org/groups/x86_64/texlive-most/}}\par
Do pracy z pakietem \LaTeX użyty został program Texstudio.\footnote{\url{http://www.texstudio.org/}}. Domyślnym kompilatorem dokumentów pozostał PdfLaTeX.\par
Podstawą do budowy dokumentu był szablon Easy Book.\footnote{\url{https://www.sharelatex.com/templates/books/easy-book}}
\paragraph{Twarde spacje.} Aby dodać twardą spację, która nie łamie się na końcach wersu - należy wstawić:
\begin{verbatim}
\,
\end{verbatim}

\paragraph{Tekst w ramce}
\begin{verbatim}
\usepackage{tcolorbox}
\begin{tcolorbox}

\end{tcolorbox}
\end{verbatim}
\begin{tcolorbox}
	Tekst na szarym tle.
\end{tcolorbox}

\chapter{Szablon dokumentu. Preambuła}

Założeniem było stworzenie dokumentu, który:
 \begin{itemize}
 \item uwzględnia specyfikę języka polskiego;
 \item pozwala na użycie listingów komend, wraz z wyróżnieniem komentarzy;
 \item używa długich tabel;
 \item pozwala na dodawania aneksu;
 \item tekst jest wyśrodkowany.
 \end{itemize}
Klasa dokumentu <<book>> pozwala na tekst wychodzący dość swobodnie poza marginesy, szczególnie przydatne jest to, kiedy używa się listingów. W przypadku utrzymania większych rygorów względem polskiej tradycji typograficznej można użyć klasy <<mwbk>>.\footnote{Więcej o spolszczeniu \LaTeX w prezentacji{\url{http://www.is.umk.pl/~zelek/kurst/kkt-w5.pdf}}}
	
Klasa <<book>> jest w prawie całkowicie zgodna z klasą <<report>>, różnicą jest np. obecność nagłówków strony w <<book>>, a ich brak w klasie <<report>>.
\begin{center}
	\begin{tabular}{|l|l|}
	\hline 
\textbf{Klasa oryginaln}a	& \textbf{Polska klasa} \\ 
	\hline 
article	&  mwart \\ 
	\hline 
report	&  mwrep \\ 
	\hline 
book	& mwbk \\ 
	\hline 
\end{tabular} 
\end{center}
\bigskip
W definicji dokumentu zostały użyte więc następujące pakiety dodatkowe:
\begin{verbatim}
\documentclass[a4paper,11pt,oneside]{book}
\usepackage[T1]{fontenc}
\usepackage[utf8]{inputenc}
\usepackage{ucs}
\usepackage{lmodern}
\usepackage{graphicx}
\usepackage{polski}
%\usepackage[polish]{babel}
\usepackage{booktabs}
\usepackage{lscape}
\usepackage{array}
\usepackage{chngpage}
\usepackage{subcaption} 
\captionsetup{compatibility=false}
\usepackage{listings}
\usepackage{underscore}
\usepackage{epstopdf}
\usepackage{longtable}
\usepackage[toc,page]{appendix}
\usepackage{hyperref}
\usepackage{pdflscape}
\usepackage{xcolor}
\lstset{basicstyle=\footnotesize\ttfamily,breaklines=true}
\lstset{framextopmargin=50pt,commentstyle=\itshape\color{purple!40!black}}
\lstset{
numbers=left,
stepnumber=1,    
firstnumber=1,
numberfirstline=false
}
\lstset{extendedchars=true,inputencoding=utf8x,literate=%
{ą}{{\k{a}}}1
{Ą}{{\k{A}}}1
{ę}{{\k{e}}}1
{Ę}{{\k{E}}}1
{ó}{{\'o}}1
{Ó}{{\'O}}1
{ś}{{\'s}}1
{Ś}{{\'S}}1
{ł}{{\l{}}}1
{Ł}{{\L{}}}1
{ż}{{\.z}}1
{Ż}{{\.Z}}1
{ź}{{\'z}}1
{Ź}{{\'Z}}1
{ć}{{\'c}}1
{Ć}{{\'C}}1
{ń}{{\'n}}1
{Ń}{{\'N}}1
}
\end{verbatim}
\section{Lualatex}
Przy użycia \textbf{lualatex} preambuła dotycząca języka jest krótsza:
\begin{verbatim}
% !TeX program = lualatex
\usepackage{polyglossia}
\setdefaultlanguage{polish}

\usepackage[
	pdfencoding=auto,% or unicode
	psdextra,
	]{hyperref}
	\hypersetup{pdfinfo={
	Title={},
	Author={Mariusz Piotrowski}
	}}
\end{verbatim}

Łatwo można manipulować fontami przez dodanie parametru:
\begin{verbatim}
\setmainfont{Iwona}
\end{verbatim}
Polskie fonty komputerowe w tradycji poligraficznej to np:\\
\begin{center}
\begin{tabular}{|c|c|}
	\hline 
\textbf{Nazwa fontu}	& \textbf{Charakterystyka} \\ 
	\hline 
	\hline 
Iwona	&  \\ 
	\hline 
Kurier	&  \\ 
	\hline 
Cyklop	&  \\ 
	\hline 
TeX Gyre Bonum	&  \\ 
	\hline 
TeX Gyre Termes	&  \\ 
	\hline 
Antykwa Torunska	&  \\ 
	\hline 
Antykwa Poltawskiego	&  \\ 
	\hline
\end{tabular} 
\end{center}
\chapter{Krótkie bibliografie}

Do generowania krótkich bibliografii można dodać do preambuły następujące komendy:
\begin{verbatim}
\usepackage{filecontents}
\begin{filecontents*}{Bibliografy.bib}
@book{}
\end{filecontents*}
\usepackage[numbers]{natbib}
\bibliographystyle{plplain}
\end{verbatim}
Plik Bibliografy.bib zostanie stworzony w katalogu i będzie przetwarzany przy kompilacji dokumentu. Pakietem stylu jest spolszczony pakiet plain <<plplain>>.\par
Na końcu dokumentu należy dodać:
\begin{verbatim}
\nocite{*}
\bibliography{Bibliografy}
\end{verbatim}
\chapter{Strona tytułowa}

Strona tytułowa posiada parametr data, który można modyfikować przez wpisanie pożądanej informacji.
Dodatkowo spisy treści dotyczą tabel, rysunków i listingów.
\begin{verbatim}
\title{\Huge \textbf{Skład raportów w \LaTeX} 
 \\ \huge Zarządzanie dokumentacją i tekstem. }
% Author
\author{\textsc{dr Mariusz Piotrowski}}
\date{\today\\wersja robocza}

\begin{document}
	
	\frontmatter %do zakomentowania w klasie raport
	\maketitle	
\tableofcontents
\listoffigures
\\lstlistoflistings	
	\mainmatter %do zakomentowania w klasie raport
\end{verbatim}

\chapter{Wstawianie grafiki}
W celu osiągnięcia wysokiej jakości pliku wynikowego należy używać grafiki w formatach wektorowych (pdf, efs). Formaty rastrowe (png, jpg)powinny być używane w rozdzielczości 300dpi. \par
Do przetwarzania plików pdf w eps można użyć programu Incscape i linii poleceń
\begin{lstlisting}[language=bash,caption={Przetwarzanie pliku pdf do eps. }]
$inkscape plik.pdf --export-eps=plik.eps
\end{lstlisting}
Następnie można osadzić plik w dokumencie.
\begin{verbatim}
\begin{figure}[h]
	\caption{Nazwa rysunku}
	\centerline{\includegraphics[width=16cm, height=4cm]{<sciezka do pliku}}
	\label{figure:1}
	\raggedbottom{Źródło: Mariusz Piotrowski, opracowanie własne}
\end{figure}
\end{verbatim}

Tak dodany rysunek będzie wycentrowany względem teksu.\par
Duże obrazy na całą stronę można dodawać w pozycji poziomej, aby uzyskać spójność czytania najłatwiej usunąć nagłówek i numer strony:
\begin{verbatim}
\begin{landscape}
	\thispagestyle{empty}
	\begin{figure}[p]
		\caption{Nazwa rysunku.}
		\centerline{\includegraphics[width=21.5cm, height=14cm]{<sciezka_do_pliku.eps}}
		%\centering
		\label{figure:2}
		\raggedbottom{Źródło: Mariusz Piotrowski, opracowanie własne}
	\end{figure} \par
\end{landscape}
\end{verbatim}
W programie TexStudio podaczas kompikowania dokumentu (pdfLatex), w którym znajduje się obraz eps pojawia się komunikat błędu. Należy wówczas zmienić w opcjach (jednorazowo) domyślny kompilator.

\chapter{Wstawianie tabel}

W celu wrzucenia wielostronicowej tabeli, z mniejszą czcionką należy zastosować następującą preambułę tabeli (dla 2 kolumn, o różnej szerokości):
\begin{verbatim}
\begin{footnotesize}
\begin{longtable}{p{1.5cm} p{10.2cm}}
\caption{}
\label{}
\endfirsthead
\hline
\multicolumn{2}{c}{Początek Tabeli}\\
Schemat                        & Nazwa tabeli\\
\hline
\endfirsthead
\hline
\multicolumn{2}{c}{Ciąg dalszy Tabeli}\\
Schemat                        & Nazwa tabeli\\
\hline
\endhead
\hline
\endfoot
\hline
\multicolumn{2}{ c }{Koniec Tabeli}\\
\hline\hline
\endlastfoot
%tu wartości
\end{longtable}
\end{footnotesize}
\end{verbatim}
Dane do tabeli przetworzyć można na stronie \url{http://www.tablesgenerator.com/}.

Mniejszą (jednostronicową) tabelę pokazuje ten kod:

\begin{verbatim}
\begin{table}[p!]
\centering
\caption{Nazwa tabeli.}
\label{table:1}
\scalebox{0.9} {
\begin{tabular}{|l|l|l|}
%tu wartości
\end{tabular}
}
\raggedbottom{Źródło: Mariusz Piotrowski, opracowanie własne}
\end{table}
\end{verbatim}


\end{document}