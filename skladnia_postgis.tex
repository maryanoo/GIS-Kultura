%%%%%%%%%%%%%%%%%%%%%%%%%%%%%%%%%%%%%%%%%%%%%%%%%%%
%% LaTeX book template                           %%
%% Author:  Amber Jain (http://amberj.devio.us/) %%
%% License: ISC license                          %%
%%%%%%%%%%%%%%%%%%%%%%%%%%%%%%%%%%%%%%%%%%%%%%%%%%%
% !TeX program = lualatex
\documentclass[a4paper,11pt,oneside]{book}
\usepackage{polyglossia}
\setdefaultlanguage{polish}
\setmainfont{Iwona}
%%%%%%%%%%%%%%%%%%%%%%%%%%%%%%%%%%%%%%%%%%%%%%%%%%%%%%%%%
% Source: http://en.wikibooks.org/wiki/LaTeX/Hyperlinks %
%%%%%%%%%%%%%%%%%%%%%%%%%%%%%%%%%%%%%%%%%%%%%%%%%%%%%%%%%
\usepackage{graphicx}
\usepackage{booktabs}
\usepackage{lscape}
\usepackage{array}
\usepackage{chngpage}
\usepackage{subcaption} 
\usepackage{listings}
\usepackage{caption}
\usepackage{underscore}
\usepackage{epstopdf}
\usepackage{longtable}
\usepackage[toc,page]{appendix}
\usepackage{hyperref}
\usepackage{pdflscape}
\usepackage{xcolor}
\lstset{basicstyle=\footnotesize\ttfamily,breaklines=true}
\lstset{framextopmargin=50pt,commentstyle=\itshape\color{gray!90!black}}
\lstset{numbers=left,stepnumber=1,firstnumber=1,numberfirstline=false}
\lstset{extendedchars=true,inputencoding=utf8x,literate=%
{ą}{{\k{a}}}1
{Ą}{{\k{A}}}1
{ę}{{\k{e}}}1
{Ę}{{\k{E}}}1
{ó}{{\'o}}1
{Ó}{{\'O}}1
{ś}{{\'s}}1
{Ś}{{\'S}}1
{ł}{{\l{}}}1
{Ł}{{\L{}}}1
{ż}{{\.z}}1
{Ż}{{\.Z}}1
{ź}{{\'z}}1
{Ź}{{\'Z}}1
{ć}{{\'c}}1
{Ć}{{\'C}}1
{ń}{{\'n}}1
{Ń}{{\'N}}1
}
\usepackage{filecontents}
\begin{filecontents*}{Bibliografy.bib}
@book{Corti:2014:PC:2616182,
	author = {Corti, Paolo and Mather, Stephen Vincent and Kraft, Thomas J and Park, Borie},
	isbn = {1849518661, 9781849518666},
	publisher = {Packt Publishing},
	title = {{PostGIS Cookbook}},
	year = {2014}
}
@book{Obe:2011:PA:2018871,
	address = {Greenwich, CT, USA},
	author = {Obe, Regina and Hsu, Leo},
	isbn = {1935182269, 9781935182269},
	publisher = {Manning Publications Co.},
	title = {{PostGIS in Action}},
	year = {2011}
}
@book{Iwanczak2013,
	address = {Gliwice},
	author = {Iwa{\'{n}}czak, Bart{\l}omiej},
	publisher = {Helion},
	title = {{Quantium GIS. Tworzenie i analiza map}},
	year = {2013}
}

\end{filecontents*}
\usepackage[numbers]{natbib}
\bibliographystyle{plplain}

%%%%%%%%%%%%%%%%%%%%%%%%%%%%%%%%%%%%%%%%%%%%%%%%%%%%%%%%%%%%%%%%%%%%%%%%%%%%%%%%
% 'dedication' environment: To add a dedication paragraph at the start of book %
% Source: http://www.tug.org/pipermail/texhax/2010-June/015184.html            %
%%%%%%%%%%%%%%%%%%%%%%%%%%%%%%%%%%%%%%%%%%%%%%%%%%%%%%%%%%%%%%%%%%%%%%%%%%%%%%%%
\newenvironment{dedication}
{
   \cleardoublepage
   \thispagestyle{empty}
   \vspace*{\stretch{1}}
   \hfill\begin{minipage}[t]{0.66\textwidth}
   \raggedright
}
{
   \end{minipage}
   \vspace*{\stretch{3}}
   \clearpage
}

%%%%%%%%%%%%%%%%%%%%%%%%%%%%%%%%%%%%%%%%%%%%%%%%
% Chapter quote at the start of chapter        %
% Source: http://tex.stackexchange.com/a/53380 %
%%%%%%%%%%%%%%%%%%%%%%%%%%%%%%%%%%%%%%%%%%%%%%%%
\makeatletter
\renewcommand{\@chapapp}{}% Not necessary...
\newenvironment{chapquote}[2][2em]
  {\setlength{\@tempdima}{#1}%
   \def\chapquote@author{#2}%
   \parshape 1 \@tempdima \dimexpr\textwidth-2\@tempdima\relax%
   \itshape}
  {\par\normalfont\hfill--\ \chapquote@author\hspace*{\@tempdima}\par\bigskip}
\makeatother

%%%%%%%%%%%%%%%%%%%%%%%%%%%%%%%%%%%%%%%%%%%%%%%%%%%
% First page of book which contains 'stuff' like: %
%  - Book title, subtitle                         %
%  - Book author name                             %
%%%%%%%%%%%%%%%%%%%%%%%%%%%%%%%%%%%%%%%%%%%%%%%%%%%

% Book's title and subtitle
\title{\Huge \textbf{Składnia SQL w danych GIS Kultura}  \\ \huge Zarządzanie danymi przestrzennymi. }
% Author
\author{\textsc{dr Mariusz Piotrowski}}
\date{\today\\wersja robocza}

\begin{document}

\frontmatter
\maketitle

%%%%%%%%%%%%%%%%%%%%%%%%%%%%%%%%%%%%%%%%%%%%%%%%%%%%%%%%%%%%%%%
% Add a dedication paragraph to dedicate your book to someone %
%%%%%%%%%%%%%%%%%%%%%%%%%%%%%%%%%%%%%%%%%%%%%%%%%%%%%%%%%%%%%%%


%%%%%%%%%%%%%%%%%%%%%%%%%%%%%%%%%%%%%%%%%%%%%%%%%%%%%%%%%%%%%%%%%%%%%%%%
% Auto-generated table of contents, list of figures and list of tables %
%%%%%%%%%%%%%%%%%%%%%%%%%%%%%%%%%%%%%%%%%%%%%%%%%%%%%%%%%%%%%%%%%%%%%%%%
\tableofcontents
\lstlistoflistings
\mainmatter

%%%%%%%%%%%
% Preface %
%%%%%%%%%%%
\chapter{Informacje wstępne}
Dokument ma charakter przeglądu typowych operacji, które wykonać można przy użyciu poleceń SQL. Polecenia dotyczą bazy Postgresql i rozszerzenia Postgis. 
 \par
Problem z projekcją układów współrzędnych, który pojawia się w analizach pokazuje film \url{https://www.youtube.com/watch?feature=youtu.be&v=kIID5FDi2JQ&app=desktop}
Dodatkowe źródła informacji o postgis:
\begin{itemize}
	\item \url{http://workshops.boundlessgeo.com/postgis-intro/index.html}
\end{itemize}

\paragraph{Bazy danych.}
W bazie \textit{ozk-gis} znajdują się następujące tabele. Są one pogrupowane w trzech schematach:. 
\begin{itemize}
	\item bazy - dane GUS i inne źródła danych o natężeniu zjawisk
	\item postgres - baza przejściowa
	\item public - dane przestrzenne i słownikowe 
\end{itemize}
\begin{footnotesize}
	\begin{longtable}{p{1.5cm} p{10.2cm}}
		\hline
		\multicolumn{2}{c}{Początek Tabeli}\\
	
		Schemat                        & Nazwa tabeli\\
		\hline
		\endfirsthead
		
		\hline
		\multicolumn{2}{c}{Ciąg dalszy Tabeli}\\
	
	Schemat                        & Nazwa tabeli\\
		\hline
		\endhead
		
		\hline
		\endfoot
		
		\hline
		\multicolumn{2}{ c }{Koniec Tabeli}\\
		\hline\hline
		\endlastfoot
bazy     & biblioteki\_2015\_gus                                               \\
bazy     & dojazdy\_praca\_nsp\_2011                                           \\
bazy     & domy\_kultury\_2015                                                 \\
bazy     & kina\_2015                                                          \\
bazy     & licea\_o\_wskazniki\_2015                                           \\
bazy     & ludnosc\_wg\_lat                                                    \\
bazy     & muzea\_2015\_gus                                                    \\
bazy     & szkoly\_ibe                                                         \\
bazy     & teatry\_2014\_it                                                    \\
bazy     & wyd\_kult\_gminy                                                    \\
bazy     & wyd\_kult\_gminy\_1osoba                                            \\
bazy     & zmiany\_teryt\_gminy                                                \\
bazy     & zmiany\_teryt\_powiaty                                              \\
postgres & dochody\_gmin\_2015\_gus                                            \\
postgres & wydatki\_gmin\_2015\_gus                                            \\
public   & area\_admin\_level\_8\_osm\_11\_2016                                \\
public   & area\_dzialki\_cemntarze\_osm\_20\_10\_2016                         \\
public   & area\_koleje\_osm\_20\_10\_2016                                     \\
public   & area\_kultura\_osm\_20\_10\_2016                                    \\
public   & area\_medycyna\_osm\_20\_10\_2016                                   \\
public   & area\_parkingi\_komunikacja\_osm\_20\_10\_2016                      \\
public   & area\_parki\_osm\_20\_10\_2016                                      \\
public   & area\_parki\_osm\_5\_11\_2016                                       \\
public   & area\_przemysl\_handel\_osm\_20\_10\_2016                           \\
public   & area\_publiczne\_straz\_policja\_osm\_20\_10\_2016                  \\
public   & area\_religijne\_osm\_20\_10\_2016                                  \\
public   & area\_rodzaje\_powierzchni\_rolnicze\_mieszkalne\_osm\_20\_10\_2016 \\
public   & area\_sport\_osm\_20\_10\_2016                                      \\
public   & area\_szkolnictwo\_osm\_20\_10\_2016                                \\
public   & area\_zabudowa\_komercyjna\_20\_10\_2016                            \\
public   & biblioteki\_mkidn\_2014                                             \\
public   & biblioteki\_wroclaw\_open                                           \\
public   & domy\_kultury\_2014                                                 \\
public   & drogi\_bdoo\_2016                                                   \\
public   & drogi\_glowne\_osm\_15\_11\_2016\_obsolete                          \\
public   & drogi\_zwykle\_osm\_18\_11\_2016                                    \\
public   & gminy                                                               \\
public   & gminy\_historyczne                                                  \\
public   & gminy\_osm\_teryt                                                   \\
public   & komunikacja-lotniska                                                \\
public   & kopalnie\_bdoo\_06\_2016                                            \\
public   & krs\_2016                                                           \\
public   & krs\_ost                                                            \\
public   & layer\_styles                                                       \\
public   & linie\_relacje\_komunikacja\_publiczna\_osm\_5\_11\_2016            \\
public   & ludnosc\_NSP2011                                                    \\
public   & miejscowosci                                                        \\
public   & obreby\_ewidencyjne                                                 \\
public   & obszar\_natura\_2000\_bdoo\_2016                                    \\
public   & obwod spisowy                                                       \\
public   & osm\_poi\_15\_11\_2016                                              \\
public   & osm\_poi\_17\_12\_2016                                              \\
public   & osm\_poi\_custom\_acces\_15\_11\_2016                               \\
public   & osm\_poi\_custom\_acces\_30\_10\_2016                               \\
public   & osm\_poi\_przystanki\_acces\_20\_10\_2016                           \\
public   & osm\_poi\_przystanki\_\_bus\_20\_10\_2016                           \\
public   & otwartezabytki\_09\_10\_2016                                        \\
public   & park\_krajobrazowy\_bdoo\_2016                                      \\
public   & park\_narodowy\_bdoo\_2016                                          \\
public   & people\_km2\_NSP                                                    \\
public   & powiaty                                                             \\
public   & rejony statystyczne                                                 \\
public   & rezerwaty\_bdoo\_06\_2016                                           \\
public   & siec\_kolejowa\_bdoo\_2016                                          \\
public   & slownik\_indeksy\_ozk                                               \\
public   & slownik\_osm\_28\_11\_2016                                          \\
public   & slownik\_ozk\_28\_11\_2016                                          \\
public   & spatial\_ref\_sys                                                   \\
public   & teatry\_danewarszawskie                                             \\
public   & temp\_krs\_99                                                       \\
public   & wody\_ladowe\_osm\_20\_10\_2016                                     \\
public   & wody\_ladowe\_powierz\_5\_11\_2016                                  \\
public   & wojewodztwa                                             		
		
\end{longtable}
\end{footnotesize}	

\begin{lstlisting}[language=SQL,caption={Nazwy tabel z bazy.}]
select * from information_schema.tables
where table_schema IN ('public', 'postgres','bazy') and table_type not like 'VIEW'
order by table_schema, table_name
\end{lstlisting}
\begin{lstlisting}[language=SQL,caption={Liczebności obiektów w gminach ze względu na typ(funkcja grupująca).}]
select count(ozk_indeks_id)as liczba, 
ozk_indeks_id,count(forma_prawna_str) as liczba_obiektow_krs, 
forma_prawna_str,slownik_ozk_28_11_2016.nazwa,gminy.jpt_nazwa_, gminy.jpt_kod_je
from krs_2016
left join  slownik_ozk_28_11_2016 
on krs_2016.ozk_indeks_id = slownik_ozk_28_11_2016.id 
inner join gminy
on st_intersects(gminy.geom, krs_2016.geom)
group by ozk_indeks_id, forma_prawna_str,slownik_ozk_28_11_2016.nazwa,gminy.jpt_nazwa_,gminy.jpt_kod_je
order by gminy.jpt_kod_je,gminy.jpt_nazwa_, ozk_indeks_id,liczba, forma_prawna_str
\end{lstlisting}

\chapter{Podstawowe operacje w SQL}		

\begin{lstlisting}[language=SQL, caption={Łaczenie gmin z danymi GUS.}]
Select gminy.geom, bazy.wyd_kult_gminy.*
from bazy.wyd_kult_gminy 
left join gminy
on bazy.wyd_kult_gminy.kod::integer = gminy.jpt_kod_je::integer
/*rzutowanie typu*/
\end{lstlisting}

\begin{lstlisting}[language=SQL, caption={Operacje matematyczne.}]
select gminy_osm_teryt.teryt, way, ("2015"::integer) - ("2005"::integer) as zmian_liczebnosc
from ludnosc_wg_lat
left join gminy_osm_teryt
on  ludnosc_wg_lat.teryt = gminy_osm_teryt.teryt
order by zmian_liczebnosc
\end{lstlisting}

\begin{lstlisting}[language=SQL, caption={Export danych do csv.}]
COPY
 (SELECT <ZAPYTANIE>) 
 TO plik.csv CSV DELIMITER ','
 /*plik zapisuje się w katalogu tmp*/
\end{lstlisting}
\chapter{Portret gminy}

Wyodrębnienie obiektów z gminy:
\begin{lstlisting}[language=SQL, caption={Obiekty z gminy.}]
Select osm.geom, osm.id, osm.nazwa, gminy.jpt_nazwa_, osm.kod_osm, osm.slownik_osm_713_nazwa_osm	
from osm_poi_custom_acces_5_11_2016 as osm
/*wewnętrzne złączenie z obiektami, które się przecinają */
Inner join public.gminy as gminy on st_intersects(osm.geom, gminy.geom)
/*wyszukanie gminy po kodzie teryt, wyodrębnienie tagu */
where gminy.jpt_kod_je ILIKE '2805011' and osm.slownik_osm_713_nazwa_osm not ILIKE '%place%'
group by slownik_osm_713_nazwa_osm, osm.geom, osm.id, osm.nazwa, gminy.jpt_nazwa_
\end{lstlisting}

Wyodrębnienie informacji o zagęszczeniu ludności w gminie:
\begin{lstlisting}[language=SQL,caption={Ludność w gminie na km$^{2}$.}]
SELECT ST_Intersection(osm.geom, gminy.geom), osm.value
from public."ludnosc_NSP2011" as osm, public.gminy as gminy
where st_intersects(osm.geom, gminy.geom) and gminy.jpt_kod_je ILIKE '2805022'
group by value, osm.id, osm.geom, gminy.geom
\end{lstlisting}
Typy obiektów OZK:
\begin{lstlisting}[language=SQL,caption={Typy obiketów OZK.}]
SELECT slownik_osm_713_nazwa_osm,
	COUNT(DISTINCT osm_poi_custom_acces_5_11_2016.id) As tot
/* wczytaj obiekty z kolumny slownik_osm_709 zlicz je wg kolumnu id(dla odrębnych wartości) i pokaż kolumnę jako tot
*/
FROM public.osm_poi_custom_acces_5_11_2016
INNER JOIN public.gminy As g
ON ST_Within(osm_poi_custom_acces_5_11_2016.geom, g.geom)
/* zrób złaczenie przestrzenne z baza gmin (nazywaj jako g) -
 złączenie według kolumn z geometrialmi przestrzennymi
*/
WHERE g.jpt_kod_je = '1201011'  AND osm_poi_custom_acces_5_11_2016.slownik_osm_713_nazwa_osm NOT ILIKE '%place%'
/* wybierz kolumnę z nazwa z gminy i znajdź tam Bolesławiec
*/
GROUP BY public.osm_poi_custom_acces_5_11_2016.slownik_osm_713_nazwa_osm
/* grupuj wg kolumny wybranej na samym początku po select
*/
ORDER BY tot DESC;
\end{lstlisting}
Obiekty w buforze 1 km od punktu
\begin{lstlisting}[language=SQL,caption={Obieky w okolicy 1km od obiektu z osm}]
SELECT ROW_NUMBER() OVER (
	PARTITION BY loc.osm_id
	ORDER BY st_distance(r.geom, loc.geom), r.nazwa) As row_num,	
	loc.nazwa, loc.id, loc.slownik_osm_713_nazwa_osm, 
	loc.geom, loc.wheelchair, loc.url, loc.weebsite, loc.opening_hours,
st_distance(r.geom, loc.geom) as dist_m
FROM osm_poi_custom_acces_5_11_2016 as loc
LEFT JOIN osm_poi_przystanki_acces_20_10_2016 as r 
	ON st_dwithin(r.geom, loc.geom, 1000)
/* buffor 1000m*/
WHERE loc.slownik_osm_713_nazwa_osm IS NOT NULL AND loc.nazwa IS NOT NULL AND r.osm_id ILIKE '%N48267476%' AND loc.slownik_osm_713_nazwa_osm NOT ilike '%place%'
\* N48267476 (przykładowy kod obiektu) wyznaczany wg kodu obiektu z bazy osm z programu osm2poi
GROUP BY r.geom, loc.geom, loc.nazwa, r.nazwa, loc.id, loc.slownik_osm_713_nazwa_osm, loc.wheelchair, loc.url, loc.weebsite, loc.opening_hours
ORDER BY dist_m, r.nazwa, row_num
\end{lstlisting}
Typy obiektów osm - zliczenie
\begin{lstlisting}[language=SQL,caption={Typy obiektów osm - zliczenie}]
SELECT 
slownik_osm_713_nazwa_osm,
COUNT(DISTINCT osm_poi_custom_acces_5_11_2016.id) As tot
/* wczytaj obiekty z kolumny slownik_osm_709 zlicz je wg kolumnu id(dla odrębnych wartości) i pokaż kolumnę jako tot
*/
FROM 
public.osm_poi_custom_acces_5_11_2016
INNER JOIN public.gminy As g
ON ST_Within(osm_poi_custom_acces_5_11_2016.geom, g.geom)
/* zrób złaczenie przestrzenne z baza gmin (nazywaj jako g) - złączenie według kolumn z geometrialmi przestrzennymi
*/
WHERE g.jpt_kod_je = '1201011'  AND osm_poi_custom_acces_5_11_2016.slownik_osm_713_nazwa_osm NOT ILIKE '%place%'
/* wybierz kolumnę z nazwa z gminy i znajdź tam Bolesławiec
*/
GROUP BY public.osm_poi_custom_acces_5_11_2016.slownik_osm_713_nazwa_osm
/* grupuj wg kolumny wybranej na samym początku po select
*/
ORDER BY tot DESC;
\end{lstlisting}
Tworzenie buffora.
\begin{lstlisting}[language=SQL,caption={Buffor 1km}]
select id, nazwa, 
	st_buffer(geom,1000) as buffer_1000
from osm_poi_custom_acces_5_11_2016 as osm
/*nazwa obiektu z bazy osm - po kodzie id*/
where osm.id = 29635
\end{lstlisting}
\begin{lstlisting}[language=SQL,caption={Obiekty wokół 1km od koordynatów geograficznych}]
select *
from krs
Where st_dwithin(geom,
/* przekształć układ z 4326 */
ST_Transform(
/* ustaw układ na 2180 */
ST_SetSRID(ST_Point(17.031924, 51.109566 ), 4326), 2180), 1000)
\end{lstlisting}

\chapter{Izochron}
\begin{lstlisting}[language=SQL,caption={Tworzenie izochronu }]

CREATE TABLE hh_2po_4pgr_node_59980_1_km AS
SELECT * FROM hh_2po_4pgr_nodes
JOIN
(SELECT * FROM pgr_drivingDistance('
SELECT id AS id,
source::int4 AS source,
target::int4 AS target,
-- koszty podróży z kolmny km
km::float8 AS cost
FROM hh_2po_4pgr',
-- id węzła od którego wyliczany jest isochron
59980,
-- odległość 
1,
false,
false)) AS hh_2po_4pgr_nodes_cost
ON
hh_2po_4pgr_nodes.id = hh_2po_4pgr_nodes_cost.id1
;
\end{lstlisting}

\begin{lstlisting}[language=SQL,caption={Tworzenie izochronu 2}]
Drop table hh_2po_4pgr_node_59980_1_ok_km;
-- tworzenie isochronu
CREATE TABLE hh_2po_4pgr_node_59980_1_ok_km AS
SELECT * FROM hh_2po_4pgr_nodes
JOIN
(SELECT * FROM pgr_drivingDistance('
SELECT id AS id,
source::int4 AS source,
target::int4 AS target,
-- koszty podróży z kolmny km
km::float8 AS cost
FROM hh_2po_4pgr',
-- id węzła od którego wyliczany jest isochron
393690,
-- odległość 
1,
false,
false)) AS hh_2po_4pgr_nodes_cost
ON
hh_2po_4pgr_nodes.id = hh_2po_4pgr_nodes_cost.id1
;
\end{lstlisting}
\begin{lstlisting}[language=SQL,caption={Otoczka wypukła}]

SELECT 
st_convexhull(ST_Collect(geom_way))
FROM hh_2po_4pgr_node_59980_1_ok_km;
\end{lstlisting}
\chapter{Podstawowe komendy}

\begin{itemize}
	\item \textbf{st_buffer}(<geometrie - projekcja 2180>,<odległość w metrach>) - tworzenie buffora;

\item\textbf{st_intersects}(<geometrie danych A - projekcja 2180>,<geometrie danych B - projekcja 2180>) - powierzchnie przecinające 
\end{itemize}
\nocite{*}

\bibliography{Bibliografy}


\end{document}